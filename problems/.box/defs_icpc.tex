\documentclass[a4paper,11pt]{article}
\usepackage{amsmath}
%\usepackage[brazil]{babel}
\usepackage[utf8]{inputenc}
\usepackage[T1]{fontenc}
\usepackage{indentfirst}
\usepackage{float}
\usepackage{fancyvrb}
\usepackage{pdftexcmds}
\usepackage{multicol}
\usepackage{amsmath}
\usepackage{hyperref}
\usepackage{ifthen}

\usepackage{epsfig}

\setlength{\marginparwidth}{0pt}
\setlength{\oddsidemargin}{-0.25cm}
\setlength{\evensidemargin}{-0.25cm}
\setlength{\marginparsep}{0pt}

\setlength{\parindent}{0cm}
\setlength{\parskip}{5pt}

\setlength{\textwidth}{16.5cm}
\setlength{\textheight}{25.5cm}

\setlength{\voffset}{-1in}

\newcommand{\insereArquivo}[1]{
\ifnum0\pdffilesize{\CWD/#1}>0
	\VerbatimInput[xleftmargin=0mm,numbers=none,obeytabs=true]{\CWD/#1}\vspace{.5em}
\fi
}

\newcommand{\theauthor}[1]
{\vspace{-2mm}\small \begin{center} \emph{Author}: #1 \end{center}\vspace{4mm}}

\newcommand{\arquivoProblema}[1]{\vspace{-0.3cm} \noindent {\em
Arquivo: \texttt{#1.[c|cpp|java]} \\}}

\newcommand{\incat}[1]{sample-#1.in}
\newcommand{\solcat}[1]{sample-#1.sol}
\newcounter{count}

\newcommand{\sampleio}{
\vspace{4pt}
\setcounter{count}{0}
\whiledo{\value{count}<10}{
  \addtocounter{count}{1}
  \IfFileExists{\CWD/\incat{\thecount}} {
  \vspace{2pt}
  \noindent
  \begin{minipage}[c]{0.95\textwidth}
  \begin{center}
  \begin{tabular}{|l|l|} \hline
  \begin{minipage}[t]{0.5\textwidth}
  \bf{\small Sample input \thecount}
  \insereArquivo{\incat{\thecount}}
  \end{minipage}
  &
  \begin{minipage}[t]{0.5\textwidth}
  \bf{\small Sample output \thecount}
  \insereArquivo{\solcat{\thecount}}
  \end{minipage}\\
  \hline
  \end{tabular}
  \end{center}
  \end{minipage} % leave next line empty

  }{} % IfFileExists
} % while

} % sampleio

\newcommand{\incluir}[1]{
\input{#1}
}

\newcommand{\inputdesc}[1]{
     \subsection*{Input}
     {#1}}

\newcommand{\inputdescline}[1]{\inputdesc{The input consists of a single line that contains {#1}}}

\newcommand{\outputdesc}[1]{
     \subsection*{Output}
     {#1}}

\newcommand{\outputdescline}[2]{\outputdesc{Output a line with {#1} representing {#2}}}

%\newcommand{\lowercaseletters}[2]{{#1} is a non-empty string of at most ${#2}$ characters;
%     each character is one of the $26$ standard lowercase letters
%     (from \literalchar{a} to \literalchar{z}).}

\newcommand{\literaltext}[1]{\mbox{``\texttt{#1}''}}
\newcommand{\literalchar}[1]{\literaltext{#1}}

\newcommand{\Ldots}[3]{{#1}, {#2}, \ldots, {#3}}
\newcommand{\iteration}[4]{${#1}=\Ldots{#2}{#3}{#4}$}

\newcommand{\outputfloatnotice}[1]{The result must be output as a rational
  number with exactly {#1} digits after the decimal point, rounded if necessary.}

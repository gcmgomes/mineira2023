% To add an image or include a .tex file you need to add
% \CWD
% to the relative (to the main document) path.
%
% Example:
% \begin{figure}
%   \centering
%   \includegraphics{\CWD/images/example.pdf}
% \end{figure}

Após inventar o avião, um dos maiores mineiros de todos os tempos, Alberto Santos-Dumont precisava testar sua invenção!

Para melhorar as chances de decolar com sucesso o seu lendário avião, o 14-bis, Santos-Dumont precisava de uma pista de decolagem que fosse bastante reta e o mais comprida possível. Uma pista de decolagem é descrita como uma sequência de números (representando a altura de cada trecho da pista) e esta é considerada reta se a diferença em valor absoluto entre duas posições adjacentes nessa sequência não é maior que um (não queremos estragar o trem de pouso da nossa querida aeronave de papel com terrenos acidentados!).

Dado um mapa topográfico da área, descrito como uma matriz NxM, imprima o tamanho da maior pista de decolagem possível, sendo que as pistas de decolagem podem ser no sentido norte-sul (colunas da matriz) ou leste-oeste (linhas da matriz).

%
% For input, use one of the following
%
\inputdesc{
  A primeira linha conterá dois números $N$ e $M$ que representam o número de linhas e colunas da matriz, respectivamente.
  
  Depois disso teremos $N$ linhas, cada uma delas contendo $M$ números inteiros $a_{i, j}$, onde $a_{i, j}$ representa o altura do terreno na linha $i$ e coluna $j$ da matriz.

  $N, M \leq 500$, $a_{i, j} \leq 10$.
}

%
% For output, use one of the following
%

\outputdesc{
  Imprima um núumero que representa o tamanho da maior pista de pouso para o 14-bis.
}

%\sampleio will look for files named sample-n.in and sample-n.sol (where n is 1, 2, 3...)
%in the documents directory and include them as samples.

\sampleio

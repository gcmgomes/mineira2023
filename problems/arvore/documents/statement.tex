% To add an image or include a .tex file you need to add
% \CWD
% to the relative (to the main document) path.
%
% Example:
% \begin{figure}
%   \centering
%   \includegraphics{\CWD/images/example.pdf}
% \end{figure}

Bacon - o Próspero - é reverenciado no reino ds capivaras por seu glorioso reinado de paz e progresso, fruto de sua inabalável devoção ao Deboismo,
religião que hoje é seguida por mais de 90\% dos habitantes da Bacônia. O Deboismo prega que todos seus seguidores devem ficar o maáximo possível
de boa na lagoa, evitando assim os estresses e aflições que diminuem a qualidade de vida e geram o caos social. Além disso, ele patrocina várias festas
populares e, sem sombra de dúvidas, as Festas da Cheia (para celebrar as chuvas de fim de ano) e suas árvore mágicas são as mais incríveis de todas.
As árvores mágicas possuem luz própria, podem ser representadas como árvores enraizadas, e funcionam da seguinte forma:

\begin{itemize}
	\item Uma sub-árvore inteira pode se acender.
	\item Uma sub-árvore inteira pode se apagar.
	\item Cada folha pode acendar ou apagar.
	\item Um nó interno da árvore está aceso se e somente se todos os seus filhos estão acesos.
\end{itemize}

Na 13ª Festa da Cheia de seu reinado, Bacon - o Próspero - levou seu bisneto, que viria a ser o rei Bacon - o Grafo - para ver a árvore mágica plantada
no palácio. Como todo bom amante da combinátoria, a pequena capivara perguntava incessantemente ao bisavô quantos nós de uma dada sub-árvore estavam acesos.
Infelizmente,  nosso herói não é lá muito bom nessa área, e pediu a você, Grande Maratonista, para o ajudar a responder as perguntas do futuro monarca!

%
% For input, use one of the following
%

\inputdesc{
	A primeira linha contém um inteiro $1 \leq N \leq 10^5$, o número de nós da árvore.
	As próximas $N-1$ linhas contém as arestas $1 \leq a_i, b_i \leq N$ da árvore. É garantido que as arestas descrevem uma árvore, que a raiz é
	sempre o vértice de número $1$, e que $a_i$ é o pai de $b_i$.
	A próxima linha contém o um inteiro $1 \leq Q \leq 10^5$ que representa o número de eventos presenciados por Bacon - o Próspero.
	As próximas $Q$ linhas contém 2 inteiros cada, $t_i, v_i$, em que $t_i \in \{1, 2, 3\}$ define o $i$-ésimo evento e $1 \leq v_i \leq N$ o vértice que enraiza a sub-árvore em questão.
	Se $t_i = 1$, então a sub-árvore enraizada em $v_i$ se acende. Se $t_i = 2$, a sub-árvore enraizada em $v_i$ se apaga.
	Se $t_i = 3$, o pequeno Bacon perguntou a seu bisavô quantos nós na sub-árvore enraizada em $v_i$ (inclusive $v_i$) estavam acesos.
}

%
% For output, use one of the following
%

\outputdesc{Para cada operação do tipo $3$, imprima o número de vértices acesos na sub-árvore definida pelo vértice dado.}

%\sampleio will look for files named sample-n.in and sample-n.sol (where n is 1, 2, 3...)
%in the documents directory and include them as samples.

\sampleio

% To add an image or include a .tex file you need to add
% \CWD
% to the relative (to the main document) path.
%
% Example:
% \begin{figure}
%   \centering
%   \includegraphics{\CWD/images/example.pdf}
% \end{figure}

GG estava pensando em um problema para a Maratona Mineira de Programação. Ele pensou no seguinte problema: é dado uma árvore (grafo não direcionado conexo acícilo) enraizada no vértice $1$, e cada vértice pode estar ligado ou desligado. Inicialmente, todos estão desligados. Existem 3 tipos de operações:

\begin{enumerate}
	\item Pintar uma sub-ávore.
	\item Despintar uma sub-árvore.
	\item Quantos vértices pintados existem em uma dada sub-árvore?
\end{enumerate}

Além disso, após as operações 1 e 2, devem ser respectivamente pintados ou despintados a quantidade mínima de vértices para que a seguinte propriedade seja satisfeita: \textbf{todo vérice não folha está pintado se, e somente se, todos os seus filhos estão pintados}.

Você, como \textit{setter} da Maratona Mineira, ficou intrigado ao ouvir sobre o problema. Porém, GG ainda não sabe como resolvê-lo eficientemente. Ajude GG e bole uma solução para o problema.

%
% For input, use one of the following
%

\inputdesc{A primeira linha contém um inteiro $1 \leq N \leq 10^5$. As próximas $N-1$ linhas contém as arestas $1 \leq a_i, b_i \leq N$ da árvore. É garantido que o grafo é uma árvore, e a raiz é o vértice $1$. A próxima linha contém o número de operações $1 \leq Q \leq 10^5$. As próximas $Q$ linhas contém 2 inteiros cada, $t, v_i$, em que $t \in \{1, 2, 3\}$ define a operação e $1 \leq v_i \leq N$ o vértice que define a sub-árvore em questão.}

%
% For output, use one of the following
%

\outputdesc{Para cada operação do tipo 3, imprima o número de vértices pintados na sub-árvore definida pelo vértice da consulta.}

%\sampleio will look for files named sample-n.in and sample-n.sol (where n is 1, 2, 3...)
%in the documents directory and include them as samples.

\sampleio

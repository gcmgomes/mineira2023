Senhor Figueiredo é um pescador famoso nos botecos da cidade de Lavras por causa das histórias duvidosas que ele conta sobre as aventuras dele no Rio Grande.
Certa vez o jovem Ceconelli, vizinho do senhor Figueiredo, resolveu investigar a veracidade dos causos.
Depois de muita insistência de Ceconelli, o senhor Figueiredo finalmente deixou que o jovem o acompanhasse, desde que carregasse os mantimentos e sua vara de pesca.
Durante a pesca, Ceconelli acabou descobrindo o segredo do senhor Figueiredo sobre as histórias do pescador; o senhor Figueiredo multiplica a quantia fisgada por um fator fixo!
Dados o fator multiplicativo usado pelo senhor Figueiredo e a quantidade real de peixes pescados, qual será a quantidade de peixes que o velho pescador irá contar para seus amigos no bar?

\section*{Entrada}

A entrada contem uma única linha com os inteiros $N$ e $K$, separados por espaço, que representam, respectivamente, a quantidade de peixes pescados e o fator usado pelo senhor Figueiredo nas histórias.

\section*{Saída}

A saída contem um único inteiro $M$, a quantidade de peixes que o senhor Figueiredo dirá que pescou.

\section*{Restrições}

\begin{itemize}
\item $1 \leq N,K \leq 1000$
\end{itemize}


\section*{Exemplos}

\exemplo

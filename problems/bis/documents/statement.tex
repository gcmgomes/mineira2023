Após inventar o avião, um dos maiores mineiros de todos os tempos, Alberto Santos-Dumont precisava testar sua invenção!

Para melhorar as chances de decolar com sucesso a sua lendária invenção, o 14-bis, Santos-Dumont precisava de uma pista de decolagem
que fosse bastante reta e o mais comprida possível. Uma pista de decolagem é descrita como uma sequência de números (representando a altura
de cada trecho da pista) e esta é considerada reta se a diferença em valor absoluto entre duas posições adjacentes nessa sequência não é maior
que um (não queremos estragar o trem de pouso da nossa querida aeronave de papel com terrenos acidentados!).

Dado um mapa topográfico da área, descrito como uma matriz $N \times M$, imprima o tamanho da maior pista de decolagem possível,
sendo que as pistas de decolagem podem ser no sentido norte-sul ou leste-oeste.

\section*{Entrada}

A primeira linha da entrada entrada contem dois inteiros $N$ e $M$, separados por espaço, que representam, respectivamente,
o número de linhas e colunas da matriz. Em seguida, seguem $N$ linhas, cada uma com $M$ inteiros, com o $j$-ésimo inteiro da $i$-ésima linha
correspondendo à célula $X_{ij}$ da matriz.

\section*{Saída}

A saída contem um único inteiro $K$, o tamanho da maior pista de pouso que podemos encontrar na matriz de entrada.

\section*{Restrições}

\begin{itemize}
    \item $1 \leq N * M \leq 10^6$
    \item $1 \leq X_{ij} \leq 10^6$
\end{itemize}


\section*{Exemplos}

\exemplo

Áurea é uma poderosa feiticeira e em seu mais novo livro, \textit{O Guia de Feitiçaria das Galaxias}, ela detalha magias 
incríveis e revolucionárias. Por exemplo, ela finalmente mostrou como transformar pedra em ouro branco e como fazer com que um animal feroz
seja gentil como uma capivara. A mágica de Áurea funciona assim: cristais coloridos são colocados ao longo de um círculo em uma ordem
específica e todos os cristais da mesma cor devem ser conectados por arcos mágicos que passam por dentro ou fora do círculo.
Além disso, arcos não podem se cruzar, pois os efeitos podem ser catastróficos e levar a uma ruptura total do espaço-tempo.
Infelizmente, o editor do \textit{Guia} fez um péssimo trabalho e apagou todos os arcos de todas figuras, sobrando assim apenas
a ordem dos cristais. Como um bom livro texto, haviam exemplos de feitiços possíveis e impossíveis, mas agora é impossível saber qual
é qual sem sua ajuda. Escreva um programa que, dada a sequência de cristais, diz se o feitiço era valido ou não.

\section*{Entrada}

A primeira linha da entrada contém um único inteiro $N$, o número de cristais no feitiço.
A segunda linha da entrada contém $N$ inteiros, separados por espaço, onde o $i$-ésimo desses inteiros $c_i$ corresponde ao $i$-ésimo cristal
do feitiço no sentido horário.

\section*{Saída}

A única linha da entrada contém um único inteiro $X$, que deve ser $1$ caso o feitiço seja válido ou $0$ caso contrário.

\section*{Restrições}

\begin{itemize}
    \item $2 \leq N \leq 1000$
    \item $1 \leq c_i \leq N$.
\end{itemize}


\section*{Exemplos}

\exemplo

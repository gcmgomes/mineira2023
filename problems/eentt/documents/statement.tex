% To add an image or include a .tex file you need to add
% \CWD
% to the relative (to the main document) path.
%
% Example:
% \begin{figure}
%   \centering
%   \includegraphics{\CWD/images/example.pdf}
% \end{figure}

Emanuel estava estudando Programação Competitiva, resolvendo problemas sobre palíndromos. Um palíndromo é uma string que permanece igual quando lida de trás para frente, como {\tt arara} e {\tt tenet}. Além disso, Emanuel sabe que uma substring de uma string $T$ é uma string que pode ser obtida apagando-se caracteres do início e fim de $T$. Por exemplo, {\tt ara}, {\tt ar} e {\tt arara} são substrings de {\tt arara}.

Ao se deparar com o problema clássico de computar quantas substring de uma string $S$ são palíndromos, Emanuel se perguntou como que esse problema poderia ser resolvido se pudessemos re-ordenar cada substring. Ou seja, dado uma string $S$, quantas substrings de $S$ (contando repetições) podem ser re-ordenadas para formar um palíndromo.

Cansado após implementar o problema clássico, Emanuel pede sua ajuda para implementar a variação para ele. Para simplificar o problema para você, ele garante que a string nunca conterá vogais ({\tt a}, {\tt e}, {\tt i}, {\tt o}, {\tt u}, {\tt y}).

%
% For input, use one of the following
%
\inputdescline{A primeira linha contém um inteiro $N$.}
\inputdescline{A segunda linha contém uma string $S$ com $N$ caracteres. $S$ não possui os caracteres {\tt a}, {\tt e}, {\tt i}, {\tt o}, {\tt u}, {\tt y}.}

%
% For output, use one of the following
%

\outputdesc{Imprima um único inteiro: o número de substrings de $S$ que podem ser re-ordenadas para formar um palíndromo.}

%\sampleio will look for files named sample-n.in and sample-n.sol (where n is 1, 2, 3...)
%in the documents directory and include them as samples.

\sampleio

\bigskip
\textbf{Explicação do Exemplo 1}: As 12 substring são:

% eennt
\begin{multicols}{2}
\begin{enumerate}
	\item {\tt e}
	\item {\tt e}
	\item {\tt n}
	\item {\tt n}
	\item {\tt t}
	\item {\tt ee}
	\item {\tt nn}
	\item {\tt een}
	\item {\tt enn}
	\item {\tt nnt}
	\item {\tt eenn}
	\item {\tt eennt}
\end{enumerate}
\end{multicols}

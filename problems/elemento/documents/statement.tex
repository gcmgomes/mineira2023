Açucar, tempero, e tudo que há de bom. Esses foram os ingredientes escolhidos pelo professor P. L. Utônio para criar a sobremessa perfeita,
porém ele acidentalmente adicionou um quarto ingrediente: o Elemento X! Diferentemente de um famoso desenho animado, a adição do Elemento X não deu certo,
e a sobremesa ficou completamente sem gosto. Como bons viajentes do tempo que somos, nossa tarefa é impedir essa catástrofe culinária.

A organização do laboratório do professor é incrivelmente eficiente e simples: os recipientes de cada um dos quatro ingredientes
é anotado com a distância entre ele e o recipiente do Elemento X. Ao chegarmos do futoro, porém, descobrimos o motivo do acidente:
o pote do Elemento X está com defeito, e mostra um valor maior que zero, o que não faz o menor sentido!
Nossa tarefa é identificar qual o recipiente danificado e informar o professor antes que um desastre culinário ocorra!

\section*{Entrada}

A entrada é composta de uma única linha contendo quatro inteiros $A, B, C, D$, cada um representando a distância do respectivo pote
para o recipiente do Elemento X.

\section*{Saída}

A saída deve conter um único inteiro $I$, que representa o índice do pote que contém o elemento $X$.

\section*{Restrições}

\begin{itemize}
\item $ 1 \leq A, B, C, D \leq 10^5$
\end{itemize}


\section*{Exemplos}

\exemplo

\begin{center}
\textit{
	Açucar, tempero, e tudo que há de bom. Esses foram os ingredientes escolhidos pelo professor P.~L.~Utônio para criar a sobremessa perfeita,
	porém ele acidentalmente adicionou um quarto ingrediente: o Elemento X! Diferentemente de um famoso desenho animado, a adição do Elemento X não deu certo,
	e a sobremesa ficou completamente sem gosto. Como bons viajantes do tempo que somos, nossa tarefa é impedir essa catástrofe culinária.
  }
\end{center}



\begin{figure}[H]
    \centering
    \includegraphics[width=6cm]{\CWD/elementox.jpg}
\end{figure}

A organização do laboratório do professor é incrivelmente eficiente e simples: o recipiente de cada um dos quatro ingredientes (colocados em ordem - $r_1, r_2, r_3, r_4$, com distância de 1 entre eles)
é anotado com a distância entre ele e o recipiente do Elemento X. Ao chegarmos do futuro, porém, descobrimos o motivo do acidente:
o pote do Elemento X está com defeito, e mostra um valor maior que zero, o que não faz o menor sentido!
Nossa tarefa é identificar qual o recipiente danificado e informar o professor antes que um desastre culinário ocorra!

\section*{Entrada}

A entrada é composta de uma única linha contendo quatro inteiros $r_1, r_2, r_3, r_4$, cada um representando a distância do respectivo pote para o recipiente do Elemento X.

\section*{Saída}

A saída deve conter um único inteiro $1 \leq I \leq 4$, que representa o índice do pote que contém o elemento $X$.

\section*{Restrições}

\begin{itemize}
\item $ 1 \leq r_1, r_2, r_3, r_4 \leq 10^5$
\end{itemize}

\section*{Exemplos}

\exemplo

\bigskip
\textbf{Explicação do Exemplo 1}: Nesse caso, o Elemento X está na primeira posição. O valor na posição 2 então é 1 porque a distancia entre o recipiente 2 e o recipiente onde o Elemento X se encontra é 1.
